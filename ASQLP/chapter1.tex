\chapter{神奇的SQL}
\section{CASE表达式}
\subsection{写在前面}
因为 CASE 表达式是不依赖于具体数据库的技术,所以可以提高 SQL 代码的可移植性。
\subsection{CASE 表达式概述}
CASE 表达式有简单 CASE 表达式(simple case expression) 和 搜 索 CASE 表 达 式(searched case expression)
两种写法
\begin{sql}{CASE表达式概述}
-- 简单CASE表达式 
CASE sex
WHEN '1' THEN '男'
WHEN '2' THEN '女'
ELSE '其他' END;

-- 搜索CASE表达式 
CASE 
WHEN sex='1' THEN '男'
WHEN sex='2' THEN '女'
ELSE '其他' END;
\end{sql}
简单 CASE 表达式正如其名,写
法简单,但能实现的事情比较有限。简单 CASE 表达式能写的条件,搜索
CASE 表达式也能写。

在编写 SQL 语句的时候需要注意,在发现为真的 WHEN 子句时,
CASE 表达式的真假值判断就会中止,而剩余的 WHEN 子句会被忽略。为了
避免引起不必要的混乱,使用 WHEN 子句时要注意条件的排他性。
\begin{sql}{WHEN子句的排他性}
-- 剩余的WHEN子句被忽略的写法示例
CASE 
WHEN COL_1 IN {'A', 'B'} THEN 'FIRSTT'
WHEN COL_1 IN {'A'} THEN 'SECOND'
     ELSE 'OTHERS' END;
\end{sql}
\catutions

\begin{enumerate}
\item 统一各分支返回的数据类型

一定要注意 CASE 表
达式里各个分支返回的数据类型是否一致。某个分支返回字符型,而其他
分支返回数值型的写法是不正确的。

\item 不要忘了写END

使用 CASE 表达式的时候,最容易出现的语法错误是忘记写 END。

\item 养成写ELSE子句的习惯

与 END 不同,ELSE 子句是可选的,不写也不会出错。不写 ELSE 子句时,
CASE 表达式的执行结果是 NULL。但是不写可能会造成“语法没有错误,结果却不对”这种不易追查原因的麻烦,所以最好明确地写上 ELSE 子句(即便
是在结果可以为 NULL 的情况下)。

\end{enumerate}


\subsection{}
\subsection{}
\subsection{}





