\chapter{Creating and Populating a Database\label{Ch02}}
\section{MySQL Data Types}
\subsection{Character Data}
Character data can be stored as either fixed-length or variable-length strings; the difference is that fixed-length strings are right-padded with spaces and always consume the same number of bytes, and variable-length strings are not right-padded with spaces and don't always consume the same number of bytes.

The maximum length for char columns is currently 255 bytes, whereas varchar columns can be up to 65,535 bytes. If you need to store longer strings (such as emails,
XML documents, etc.), then you will want to use one of the text types (mediumtext
and longtext).

\begin{tcolorbox}
    An exception is made in the use of varchar for Oracle Database. Oracle users should use the varchar2 type when defining variable-length character columns.
\end{tcolorbox}

\subsubsection*{Character sets}

You may choose to use a different character set for each character column in your database, and you can even store different character sets within the same table. To choose a character set other than the default when defining a column, simply name one of the supported character sets after the type definition.

\textsf{varchar(20) character set latin1}

With MySQL, you may also set the default character set for your entire database:

\textsf{create database european\_sales character set latin1;}

\subsubsection*{Text data}
If you need to store data that might exceed the 64 KB limit for varchar columns, you will need to use one of the text types.
\begin{table}
    \centering
    \caption{MySQL text types}
    \label{tbl2-1}
    \begin{tabular}{lr}
        \hline
        Text type  & Maximum number of bytes \\
        \hline
        tinytext   & 255                     \\
        text       & 65,535                  \\
        mediumtext & 16,777,215              \\
        longtext   & 4,294,967,295           \\
        \hline
    \end{tabular}
\end{table}

When choosing to use one of the text types, you should be aware of the following:
\begin{itemize}
    \item If the data being loaded into a text column exceeds the maximum size for that type, the data will be truncated.
    \item Trailing spaces will not be removed when data is loaded into the column.
    \item When using text columns for sorting or grouping, only the first 1,024 bytes are used, although this limit may be increased if necessary.
\end{itemize}

\section{Table Creation}
\section{Populating and Modifying Tables}
\subsection{Inserting Data}
There are three main components to an insert statement:
\begin{itemize}
    \item The name of the table into which to add the data
    \item The names of the columns in the table to be populated
    \item The values with which to populate the columns
\end{itemize}
\subsubsection*{Generating numeric key data}
\begin{tcolorbox}
    If you are running these statements in your database, you will first need to disable the foreign key constraint on the table, and then re-enable the constraints when finished. The progression of statements would be:

    \textsf{set foreign\_key\_checks=0;}
\end{tcolorbox}
