\chapter{Filtering\label{Ch04}}
\section{Range Conditions}
\subsection*{The between operator}
When you have both an upper and lower limit for your range, you may choose to use a single condition that utilizes the between operator rather than using two separate conditions.

\subsection*{String ranges}
To work with string ranges, you need to know the order of the characters within your character set (the order in which the characters within a character set are sorted is called a collation(排序规则)).
\section{Membership Conditions}
In some cases, you will not be restricting an expression to a single value or range of values but rather to a finite set of values.

\subsection*{Using subqueries}
Along with writing your own set of expressions, such as (`G',`PG'), you can use a subquery to generate a set for you on the fly.
\subsection*{Using not in}
Sometimes you want to see whether a particular expression exists within a set of expressions, and sometimes you want to see whether the expression does not exist within the set.
\section{Matching Conditions}
\subsection*{Using wildcards}
\begin{table}
    \centering
    \caption{Wildcard characters}
    \begin{tabular}{ll}
        \hline
        Wildcard character & Matches                                \\
        \hline
        \_                 & Exactly one character                  \\
        \%                 & Any number of characters (including 0) \\
        \hline
    \end{tabular}
\end{table}
When building conditions that utilize search expressions, you use the like operator.

\begin{table}
    \centering
    \caption{Sample search expressions}
    \begin{tabular}{ll}
        \hline
        Search expression    & Interpretation                                                       \\
        \hline
        F\%                  & Strings beginning with F                                             \\
        \%t                  & Strings ending with t                                                \\
        \%bas\%              & Strings containing the substring 'bas'                               \\
        \_\_t\_              & Four-character strings with a t in the third position                \\
        \_\_\_-\_\_-\_\_\_\_ & 11-character strings with dashes in the fourth and seventh positions \\
        \hline
    \end{tabular}
\end{table}
\section{Null: That Four-Letter Word}
null is a bit slippery, however, as there are various flavors of null:

\textit{Not applicable}: Such as the employee ID column for a transaction that took place at an ATM machine

\textit{Value not yet known}: Such as when the federal ID is not known at the time a customer row is created

\textit{Value undefined}: Such as when an account is created for a product that has not yet been added to the database

When working with null, you should remember:
\begin{itemize}
    \item An expression can be null, but it can never equal null.
    \item Two nulls are never equal to each other.
\end{itemize}

When working with a database that you are not familiar with, it is a good idea to find out which columns in a table allow nulls so that you can take appropriate measures with your filter conditions to keep data from slipping through the cracks.