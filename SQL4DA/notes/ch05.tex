\chapter{\label{ch05}}
\section{Window Functions}
\subsection{The Basics of Window Functions}
\figures{fig5-6}{Windows for customers using COUNT(*) ordered by the customer\_id window query}

\autoref{fig5-6} the dataset is ordered using customer\_id, which happens to be the primary key. As such each row has a unique value and forms a value group. The first value group, without any row before it, forms its own window, which contains only the first row. The second value group's window will contain both itself and the row before it, which means the first and second row. Then the third value group's window will contain itself and the two rows before it, and so on and so forth. Every value group has its window. Once the windows are established, for every value group, the window function is calculated based on the window. In this example, this means COUNT is applied to every window. Thus, value group 1 (the first row) gets 1 as the result since its Window 1 contains one row, value group 2 (the second row) gets 2 since its Window 2 contains two rows, and so on and so forth. The results are applied to every row in this value group if the group contains multiple rows. Note that the window is used for calculation only. The results are assigned to rows in the value group, not assigned to the rows in the window.