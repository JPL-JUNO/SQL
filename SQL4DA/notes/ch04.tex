\chapter{Aggregate Functions for Data Analysis\label{Ch04}}
\section{Aggregate Functions}
\begin{table}
    \centering
    \begin{tabularx}{\textwidth}{lX}
        \hline
        Function                          & Explanation                                                                                                           \\
        \hline
        count(columnX)                    & Counts the number of rows in columnX that have a non-NULL value                                                       \\
        count(*)                          & Counts the number of rows in the output table                                                                         \\
        min(columnX)                      & Returns the minimum value in columnX. For text columns, it returns the value that would appear first alphabetically   \\
        max(columnX)                      & Returns the maximum value in columnX                                                                                  \\
        sum(columnX)                      & Returns the sum of all values in columnX                                                                              \\
        avg(columnX)                      & Returns the average of all values in columnX                                                                          \\
        stddev(columnX)                   & Returns the samples standard deviation of all values in columnX                                                       \\
        var(columnX)                      & Returns the samples variance of all values in columnX                                                                 \\
        regr\_slope(columnX, columnY)     & Returns the slope of linear regression for columnX as the response variable and columnY as the predictor variable     \\
        regr\_intercept(columnX, columnY) & Returns the intercept of linear regression for columnX as the response variable and columnY as the predictor variable \\
        corr(columnX, columnY)            & Calculates the Pearson correlation between columnX and columnY in the data                                            \\
        \hline
    \end{tabularx}
\end{table}